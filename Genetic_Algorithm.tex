\documentclass{ctexart}
\usepackage{geometry} %控制页边距等
\usepackage{caption}
\usepackage{titlesec}
\usepackage{subcaption}
\usepackage{xcolor}   %定义新颜色,以及与颜色有关的一些操作
\usepackage{enumitem} %用来创建圆形实心项目符号
\usepackage{minted}
\usepackage{xcolor}

% 设置代码样式,如背景色、字体大小等
\definecolor{LightGray}{gray}{0.9}
\newminted{python}{
    frame=lines,
    framesep=2mm,
    bgcolor=LightGray,
    fontsize=\small
}

\geometry{a4paper, margin=0.75in} 
\pagestyle{plain}

\title{Genetic_Algorithm}
\author{} 
\date{} 

\begin{document}
	\large
	\maketitle
	\vspace{-50pt}
	\section{实验原理}
	\subsection{遗传算法}
	遗传算法(Genetic Algorithm, GA)是一种受到生物进化启发的全局优化搜索算法,由美国学者约翰·霍兰德(John Holland)于20世纪70年代提出。它通过模拟自然选择和遗传机制,在解空间中搜索最优解。遗传算法的主要特点是直接对结构对象进行操作,不需要梯度信息,具有内在的并行性和全局搜索能力。
	\subsection{排课问题描述}
	课程表生成是一个典型的组合优化问题,需要在满足一系列约束条件下,为课程、教师、教室和时间分配合理的组合。具体约束包括:
	\vspace{-10pt}
	\subsubsection{硬约束}
	\begin{itemize}[label=\textbullet]
		\item \textbf{时间约束:}同一教师不能在同一时间教多门课程,同一教室不能在同一时间容纳多门课程,同一班级不能在同一时间上多门课程。
		\item \textbf{空间约束:}每门课程需要分配到一个教室,教室资源有限。
	\end{itemize}
	\vspace{-20pt}
	\subsubsection{软约束}
	\begin{itemize}[label=\textbullet]
		\item 课程应合理分布在不同时间段,避免过度集中或分散。
	\end{itemize}
	
	\subsection{解题逻辑}
	\subsubsection{编码}
	将课程安排表示为一个个体(染色体),每个基因代表一门课程的安排,包括班级、课程ID、教师ID、教室、星期和时间段。
	\subsubsection{初始种群生成}
	随机生成500个个体,构成初始种群。每个个体代表一种可能的课程安排方案。
	\subsubsection{适应度函数}
	定义适应度函数来评估每个个体的优劣。适应度函数综合考虑了时间冲突、课程分布合理性等因素。
	\begin{verbatim}
		fitness = 1 / (conflict_count + 1)
		fitness += course_spread * 0.01
		fitness += class_spread * 0.01
	\end{verbatim}
	\vspace{-20pt}
	其中 conflict\_count (硬约束) 为教师、地点、时间冲突的总数,加一是防止除零。course\_spread 和 
	class\_spread (软约束) 分别是同一课程在周内的分布间隔天数和同一天内课程间隔时间,权值均为0.01。
	\subsection{选择操作}
	采用锦标赛选择(Tournament Selection)(每轮锦标赛挑选7个个体),从当前种群中选择适应度较高的个体作为父代,用于产生下一代。
	\subsection{交叉操作}
	以 0.9 的概率(交叉率)对两个父代个体进行交叉,生成新的子代个体。交叉过程中,确保新生成的课程安排满足基本的时间和空间约束。
	\subsection{变异操作}
	以 0.01 的概率(变异率)对个体中的某些基因进行变异,改变课程的教师、教室、时间等属性,增加种群的多样性。
	\subsection{进化流程(主循环)}
	重复选择、交叉和变异操作,不断进化种群,直到达到预设的代数(10000代),或者当适应度变化不超过0.01。每100代输出一次当前最大适应度值。最佳答案的适应度不能小于整个过程中的最大适应度(即结束时如果小于过程中最大值,则选择过程中最大值作为最佳结果。)
	\subsection{进化结果}
	进化结束后,输出最佳适应度对应的课程安排方案,并以表格形式展示。
	
	\vspace{100pt}
	\section{结果展示}
	\subsection{适应度}
	停止在第 7888 代
		\begin{table}[htbp]
		\centering
		\caption{每100轮次最大适应度值}
		\begin{tabular}{|l|l||l|l|}
		\hline
		\textbf{Generation} & \textbf{Best Fitness} & \textbf{Generation} & \textbf{Best Fitness} \\
		\hline
		100 & 1.2100 & 3700 & 1.2200 \\
		\hline
		200 & 1.1600 & 3800 & 1.2100 \\
		\hline
		300 & 1.1600 & 3900 & 1.2100 \\
		\hline
		400 & 0.7000 & 4000 & 1.1600 \\
		\hline
		500 & 0.6300 & 4100 & 1.1500 \\
		\hline
		600 & 0.4833 & 4200 & 1.1000 \\
		\hline
		700 & 1.2000 & 4300 & 0.6800 \\
		\hline
		800 & 0.6400 & 4400 & 1.2100 \\
		\hline
		900 & 1.2100 & 4500 & 1.1900 \\
		\hline
		1000 & 1.1400 & 4600 & 1.2000 \\
		\hline
		1100 & 1.2100 & 4700 & 0.6900 \\
		\hline
		1200 & 1.1700 & 4800 & 1.1600 \\
		\hline
		1300 & 1.2200 & 4900 & 1.1400 \\
		\hline
		1400 & 0.6500 & 5000 & 1.2200 \\
		\hline
		1500 & 1.1900 & 5100 & 1.1500 \\
		\hline
		1600 & 1.1600 & 5200 & 1.2100 \\
		\hline
		1700 & 1.1400 & 5300 & 0.5233 \\
		\hline
		1800 & 1.1800 & 5400 & 1.1300 \\
		\hline
		1900 & 1.2100 & 5500 & 1.1500 \\
		\hline
		2000 & 0.6700 & 5600 & 1.2300 \\
		\hline
		2100 & 1.1400 & 5700 & 1.1900 \\
		\hline
		2200 & 1.1600 & 5800 & 1.2300 \\
		\hline
		2300 & 0.7100 & 5900 & 1.1900 \\
		\hline
		2400 & 1.2000 & 6000 & 1.1600 \\
		\hline
		2500 & 1.1500 & 6100 & 0.6600 \\
		\hline
		2600 & 1.1600 & 6200 & 1.2000 \\
		\hline
		2700 & 0.6500 & 6300 & 1.1600 \\
		\hline
		2800 & 1.2300 & 6400 & 1.1800 \\
		\hline
		2900 & 1.2200 & 6500 & 1.1500 \\
		\hline
		3000 & 1.1900 & 6600 & 1.2200 \\
		\hline
		3100 & 1.1900 & 6700 & 1.2200 \\
		\hline
		3200 & 1.1900 & 6800 & 1.2300 \\
		\hline
		3300 & 1.1900 & 6900 & 1.2100 \\
		\hline
		3400 & 1.1100 & 7000 & 1.1900 \\
		\hline
		3500 & 1.2000 & 7100 & 1.1700 \\
		\hline
		3600 & 0.6600 & 7200 & 1.1800 \\
		\hline
		\end{tabular}
	\end{table}
	\newpage
	\subsection{最后结果}
	最大适应度值:1.32
	\begin{table}[htbp]
		\centering
		\caption{最佳排课方案}
		\begin{tabular}{|l|l|l|l|l|l|}
		\hline
		\textbf{班级} & \textbf{课程} & \textbf{教师} & \textbf{教室} & \textbf{星期} & \textbf{时间} \\
		\hline
		1班 & 概率统计 & 赵思 & 2101 & Mon & 8:00-9:50 \\
		\hline
		1班 & 计算机应用基础 & 赵宏生 & 2103 & Mon & 10:10-12:00 \\
		\hline
		1班 & 运筹学 & 童仁兴 & 2103 & Mon & 14:00-15:50 \\
		\hline
		1班 & 人工智能基础 & 项峰炎 & 2101 & Mon & 19:30-21:20 \\
		\hline
		1班 & 数字图像处理 & 项健德 & 2103 & Tue & 16:10-18:00 \\
		\hline
		1班 & 嵌入式智能系统 & 项友志 & 2103 & Wed & 10:10-12:00 \\
		\hline
		1班 & 高等数学 & 郑军锦 & 2103 & Thu & 19:30-21:20 \\
		\hline
		1班 & 线性代数 & 冯桥 & 2103 & Fri & 14:00-15:50 \\
		\hline
		1班 & 计算机应用基础 & 赵宏生 & 2102 & Fri & 16:10-18:00 \\
		\hline
		2班 & 数字图像处理 & 项健德 & 2103 & Mon & 16:10-18:00 \\
		\hline
		2班 & 人工智能基础 & 周友国 & 2102 & Wed & 10:10-12:00 \\
		\hline
		2班 & 运筹学 & 童仁兴 & 2102 & Wed & 19:30-21:20 \\
		\hline
		2班 & 概率统计 & 赵思 & 2101 & Thu & 10:10-12:00 \\
		\hline
		2班 & 高等数学 & 冯桥 & 2101 & Thu & 14:00-15:50 \\
		\hline
		2班 & 嵌入式智能系统 & 孙文 & 2101 & Fri & 8:00-9:50 \\
		\hline
		2班 & 微机原理与接口技术 & 董南振 & 2102 & Fri & 14:00-15:50 \\
		\hline
		2班 & 线性代数 & 郑军锦 & 2103 & Fri & 16:10-18:00 \\
		\hline
		2班 & 微机原理与接口技术 & 董友裕 & 2102 & Sat & 8:00-9:50 \\
		\hline
		3班 & 数据结构 & 李兴 & 2101 & Mon & 14:00-15:50 \\
		\hline
		3班 & 运筹学 & 童仁兴 & 2101 & Mon & 16:10-18:00 \\
		\hline
		3班 & 数据结构 & 李兴 & 2103 & Tue & 8:00-9:50 \\
		\hline
		3班 & 人工智能基础 & 赵荣正 & 2103 & Tue & 19:30-21:20 \\
		\hline
		3班 & 数字图像处理 & 郑宏文 & 2102 & Wed & 8:00-9:50 \\
		\hline
		3班 & 高等数学 & 冯桥 & 2101 & Thu & 8:00-9:50 \\
		\hline
		3班 & 线性代数 & 郑军锦 & 2101 & Fri & 10:10-12:00 \\
		\hline
		3班 & 概率统计 & 赵思 & 2101 & Fri & 19:30-21:20 \\
		\hline
		3班 & 嵌入式智能系统 & 祝桥 & 2103 & Sat & 8:00-9:50 \\
		\hline
		\end{tabular}
	\end{table}
	
	\newpage
	\section{代码}
	\begin{pythoncode}
	import random
	import numpy as np
	from prettytable import PrettyTable
	from torch.utils.tensorboard import SummaryWriter  # 引入TensorBoard
	
	# 输入数据
	classrooms = ["2101", "2102", "2103"]
	classes = ["1班", "2班", "3班"]
	courses = {
	    "52005": "高等数学",
	    "53202": "线性代数",
	    "53215": "概率统计",
	    "53230": "运筹学",
	    "54007": "计算机应用基础",
	    "54200": "微机原理与接口技术",
	    "54459": "数据结构",
	    "54569": "数字图像处理",
	    "54830": "人工智能基础",
	    "54976": "嵌入式智能系统"
	}
	teachers = {
	    "T1": {"name": "冯桥", "courses": ["52005", "53202"]},
	    "T2": {"name": "赵思", "courses": ["53215"]},
	    "T3": {"name": "赵宏生", "courses": ["54007"]},
	    "T4": {"name": "赵国", "courses": ["54007"]},
	    "T5": {"name": "董南振", "courses": ["53230", "54200"]},
	    "T6": {"name": "祝毅勇", "courses": ["53230"]},
	    "T7": {"name": "李兴", "courses": ["54459", "54569"]},
	    "T8": {"name": "赵荣正", "courses": ["54830"]},
	    "T9": {"name": "王磊宏", "courses": ["54976"]},
	    "T10": {"name": "周友国", "courses": ["54830"]},
	    "T11": {"name": "孙文", "courses": ["54976"]},
	    "T12": {"name": "项峰炎", "courses": ["54830"]},
	    "T13": {"name": "项健德", "courses": ["54569", "54976"]},
	    "T14": {"name": "李艺祥", "courses": ["54830", "54976"]},
	    "T15": {"name": "祝桥", "courses": ["54830", "54976"]},
	    "T16": {"name": "郑宏文", "courses": ["54569", "54976"]},
	    "T17": {"name": "赵国健", "courses": ["54976"]},
	    "T18": {"name": "郑军锦", "courses": ["52005", "53202"]},
	    "T19": {"name": "董友裕", "courses": ["53230", "54200"]},
	    "T20": {"name": "童仁兴", "courses": ["53230"]},
	    "T21": {"name": "陈南坚", "courses": ["54459"]},
	    "T22": {"name": "项友志", "courses": ["54200", "54976"]}
	}
	schedule_requirements = {
	    "1班": {
	        "52005": 1,
	        "53202": 1,
	        "53215": 1,
	        "53230": 1,
	        "54007": 2,
	        "54200": 0,
	        "54459": 0,
	        "54569": 1,
	        "54830": 1,
	        "54976": 1
	    },
	    "2班": {
	        "52005": 1,
	        "53202": 1,
	        "53215": 1,
	        "53230": 1,
	        "54007": 0,
	        "54200": 2,
	        "54459": 0,
	        "54569": 1,
	        "54830": 1,
	        "54976": 1
	    },
	    "3班": {
	        "52005": 1,
	        "53202": 1,
	        "53215": 1,
	        "53230": 1,
	        "54007": 0,
	        "54200": 0,
	        "54459": 2,
	        "54569": 1,
	        "54830": 1,
	        "54976": 1
	    }
	}
	time_slots = {
	    1: {"start": "8:00", "end": "9:50"},
	    2: {"start": "10:10", "end": "12:00"},
	    3: {"start": "14:00", "end": "15:50"},
	    4: {"start": "16:10", "end": "18:00"},
	    5: {"start": "19:30", "end": "21:20"}
	}
	days = ["Mon", "Tue", "Wed", "Thu", "Fri", "Sat"]
	
	# 遗传算法参数
	POPULATION_SIZE = 500  # 增加种群大小
	MUTATION_RATE = 0.01  # 调整变异率
	CROSSOVER_RATE = 0.7  # 调整交叉率
	GENERATIONS = 10000  # 增加最大代数
	TOURNAMENT_SIZE = 7  # 调整锦标赛选择的大小
	ELITE_SIZE = 2  # 精英保留策略,保留前2个优质个体
	
	
	# 课程安排类
	class CourseArrangement:
	    def __init__(self, class_name, course_id, teacher_id, classroom, day, time_slot):
	        self.class_name = class_name
	        self.course_id = course_id
	        self.teacher_id = teacher_id
	        self.classroom = classroom
	        self.day = day
	        self.time_slot = time_slot
	
	    def __str__(self):
	        return f"{self.class_name} {courses[self.course_id]} \
		        {teachers[self.teacher_id]['name']}{self.classroom} {days[self.day - 1]} \
		        {time_slots[self.time_slot]['start']}-{time_slots[self.time_slot]['end']}"
	
	
	# 生成初始种群
	def generate_initial_population():
	    population = []
	    for _ in range(POPULATION_SIZE):
	        schedule = []
	        for class_name in classes:
	            for course_id, weekly_hours in schedule_requirements[class_name].items():
	                for _ in range(weekly_hours):
	                    possible_teachers = [t \
	                    for t in teachers if course_id in teachers[t]["courses"]]
	                    teacher_id = random.choice(possible_teachers)
	                    classroom = random.choice(classrooms)
	                    day = random.randint(1, 6)
	                    time_slot = random.randint(1, 5)
	                    schedule.append(CourseArrangement(class_name, course_id, teacher_id, \
	                                                      classroom, day, time_slot))
	        population.append(schedule)
	    return population
	
	
	# 计算适应度
	def calculate_fitness(individual):
	    conflict_count = 0
	    course_spread = 0
	    class_spread = 0
	
	    teacher_schedule = {t: set() for t in teachers}
	    for arr in individual:
	        time_key = (arr.day, arr.time_slot)
	        if time_key in teacher_schedule[arr.teacher_id]:
	            conflict_count += 1
	        else:
	            teacher_schedule[arr.teacher_id].add(time_key)
	
	    classroom_schedule = {c: set() for c in classrooms}
	    for arr in individual:
	        time_key = (arr.day, arr.time_slot)
	        if time_key in classroom_schedule[arr.classroom]:
	            conflict_count += 1
	        else:
	            classroom_schedule[arr.classroom].add(time_key)
	
	    class_schedule = {cls: set() for cls in classes}
	    for arr in individual:
	        time_key = (arr.day, arr.time_slot)
	        if time_key in class_schedule[arr.class_name]:
	            conflict_count += 1
	        else:
	            class_schedule[arr.class_name].add(time_key)
	
	    course_days = {}
	    for arr in individual:
	        if (arr.class_name, arr.course_id) not in course_days:
	            course_days[(arr.class_name, arr.course_id)] = set()
	        course_days[(arr.class_name, arr.course_id)].add(arr.day)
	
	    for key in course_days:
	        days_used = sorted(course_days[key])
	        for i in range(1, len(days_used)):
	            course_spread += (days_used[i] - days_used[i - 1] - 1)
	
	    class_day_counts = {cls: [0] * 7 for cls in classes}
	    for arr in individual:
	        class_day_counts[arr.class_name][arr.day] += 1
	
	    for cls in classes:
	        avg = sum(class_day_counts[cls][1:7]) / 6
	        for day in range(1, 7):
	            class_spread += abs(class_day_counts[cls][day] - avg)
	
	    fitness = 1 / (conflict_count + 1)
	    fitness += course_spread * 0.01
	    fitness += class_spread * 0.01
	
	    return fitness
	
	
	# 选择操作(锦标赛选择)
	def select_parent(population):
	    selected = random.sample(population, TOURNAMENT_SIZE)
	    return max(selected, key=lambda x: x[1])
	
	
	# 交叉操作
	def crossover(parent1, parent2):
	    if random.random() > CROSSOVER_RATE:
	        return parent1
	
	    child = []
	    for i in range(len(parent1)):
	        if random.random() < 0.5:
	            child.append(parent1[i])
	        else:
	            child.append(parent2[i])
	
	    # 冲突解决:教师时间冲突
	    teacher_schedule = {t: set() for t in teachers}
	    for arr in child:
	        time_key = (arr.day, arr.time_slot)
	        attempt = 0
	        max_attempts = 100  # 设置最大尝试次数
	        while time_key in teacher_schedule[arr.teacher_id] and attempt < max_attempts:
	            arr.day = random.randint(1, 6)
	            arr.time_slot = random.randint(1, 5)
	            time_key = (arr.day, arr.time_slot)
	            attempt += 1
	        teacher_schedule[arr.teacher_id].add(time_key)
	
	    # 冲突解决:教室时间冲突
	    classroom_schedule = {c: set() for c in classrooms}
	    for arr in child:
	        time_key = (arr.day, arr.time_slot)
	        attempt = 0
	        while time_key in classroom_schedule[arr.classroom] and attempt < max_attempts:
	            arr.classroom = random.choice(classrooms)
	            time_key = (arr.day, arr.time_slot)
	            attempt += 1
	        classroom_schedule[arr.classroom].add(time_key)
	
	    # 冲突解决:班级时间冲突
	    class_schedule = {cls: set() for cls in classes}
	    for arr in child:
	        time_key = (arr.day, arr.time_slot)
	        attempt = 0
	        while time_key in class_schedule[arr.class_name] and attempt < max_attempts:
	            arr.day = random.randint(1, 6)
	            arr.time_slot = random.randint(1, 5)
	            time_key = (arr.day, arr.time_slot)
	            attempt += 1
	        class_schedule[arr.class_name].add(time_key)
	
	    return child
	
	
	# 变异操作
	def mutate(individual):
	    for i in range(len(individual)):
	        if random.random() < MUTATION_RATE:
	            possible_teachers = [t for t in teachers \
	                                 if individual[i].course_id in teachers[t]["courses"]]
	            individual[i].teacher_id = random.choice(possible_teachers)
	            individual[i].classroom = random.choice(classrooms)
	            individual[i].day = random.randint(1, 6)
	            individual[i].time_slot = random.randint(1, 5)
	
	            # 冲突解决:教师时间冲突
	            teacher_schedule = {t: set() for t in teachers}
	            time_key = (individual[i].day, individual[i].time_slot)
	            attempt = 0
	            max_attempts = 100  # 设置最大尝试次数
	            while time_key in teacher_schedule[individual[i].teacher_id] \
	                    and attempt < max_attempts:
	                individual[i].day = random.randint(1, 6)
	                individual[i].time_slot = random.randint(1, 5)
	                time_key = (individual[i].day, individual[i].time_slot)
	                attempt += 1
	            teacher_schedule[individual[i].teacher_id].add(time_key)
	
	            # 冲突解决:教室时间冲突
	            classroom_schedule = {c: set() for c in classrooms}
	            time_key = (individual[i].day, individual[i].time_slot)
	            attempt = 0
	            while time_key in classroom_schedule[individual[i].classroom] \
	                    and attempt < max_attempts:
	                individual[i].classroom = random.choice(classrooms)
	                time_key = (individual[i].day, individual[i].time_slot)
	                attempt += 1
	            classroom_schedule[individual[i].classroom].add(time_key)
	
	            # 冲突解决:班级时间冲突
	            class_schedule = {cls: set() for cls in classes}
	            time_key = (individual[i].day, individual[i].time_slot)
	            attempt = 0
	            while time_key in class_schedule[individual[i].class_name] \
	                    and attempt < max_attempts:
	                individual[i].day = random.randint(1, 6)
	                individual[i].time_slot = random.randint(1, 5)
	                time_key = (individual[i].day, individual[i].time_slot)
	                attempt += 1
	            class_schedule[individual[i].class_name].add(time_key)
	
	    return individual
	
	# 遗传算法主循环
	def genetic_algorithm():
	    population = generate_initial_population()
	    fitness_scores = [calculate_fitness(ind) for ind in population]
	
	    # 初始化TensorBoard
	    writer = SummaryWriter('runs/scheduling_experiment')
	
	    # 用于保存每100轮的最大适应度值
	    max_fitness_every_100 = []
	
	    # 记录过程中最大适应度值及其对应的个体
	    max_fitness = max(fitness_scores)
	    max_fitness_individual = population[np.argmax(fitness_scores)]
	
	    # 用于判断适应度值波动是否小于等于0.01
	    fitness_history = [max_fitness]
	
	    for generation in range(GENERATIONS):
	        # 每100代输出一次进度
	        if (generation + 1) % 100 == 0:
	            best_index = np.argmax(fitness_scores)
	            print(
	                f"Generation {generation + 1}, Best Fitness: \
	                {fitness_scores[best_index]:.4f}, Remaining Generations: \
	                {GENERATIONS - (generation + 1)}")
	            max_fitness_every_100.append((generation + 1, fitness_scores[best_index]))
	
	        # 记录每一代的最佳适应度值到TensorBoard
	        current_best_fitness = max(fitness_scores)
	        writer.add_scalar('Best Fitness', current_best_fitness, generation)
	
	        # 更新过程中最大适应度值及其对应的个体
	        if current_best_fitness > max_fitness:
	            max_fitness = current_best_fitness
	            max_fitness_individual = population[np.argmax(fitness_scores)]
	
	        # 添加到适应度历史记录
	        fitness_history.append(current_best_fitness)
	        if len(fitness_history) > 10:  # 保留最近10代的适应度值
	            fitness_history.pop(0)
	
	        # 计算适应度值的波动
	        if len(fitness_history) >= 10:
	            fitness_std = np.std(fitness_history)
	            if fitness_std <= 0.01:
	                print(f"Adaptation value fluctuation is less than or equal to 0.01, \
	                stop at generation {generation + 1}")
	                break
	
	        # 精英保留策略:保留前ELITE_SIZE个优质个体
	        elite_indices = np.argsort(fitness_scores)[-ELITE_SIZE:]
	        elite_individuals = [population[i] for i in elite_indices]
	
	        # 选择下一代
	        new_population = elite_individuals.copy()  # 添加精英个体到新种群
	        while len(new_population) < POPULATION_SIZE:
	            parent1 = select_parent(list(zip(population, fitness_scores)))[0]
	            parent2 = select_parent(list(zip(population, fitness_scores)))[0]
	            child = crossover(parent1, parent2)
	            child = mutate(child)
	            new_population.append(child)
	
	        population = new_population
	        fitness_scores = [calculate_fitness(ind) for ind in population]
	
	        # 如果达到10000轮次,停止
	        if generation + 1 >= 10000:
	            print("Reached 10000 generations, stop")
	            break
	
	    # 关闭TensorBoard
	    writer.close()
	
	    # 将每100轮的最大适应度值保存到文件
	    with open('max_fitness_every_100.txt', 'w') as f:
	        for gen, fitness in max_fitness_every_100:
	            f.write(f"Generation {gen}, Best Fitness: {fitness:.4f}\n")
	
	    # 确保最终答案的适应度不小于过程中最大适应度
	    final_best_index = np.argmax(fitness_scores)
	    final_best_fitness = fitness_scores[final_best_index]
	    if final_best_fitness >= max_fitness:
	        best_individual = population[final_best_index]
	    else:
	        best_individual = max_fitness_individual
	
	    return best_individual, max_fitness, max_fitness_individual
	
	
	# 格式化输出课表
	def format_schedule(schedule):
	    sorted_schedule = sorted(schedule, key=lambda x: (x.class_name, x.day, x.time_slot))
	
	    table = PrettyTable()
	    table.field_names = ["班级", "课程", "教师", "教室", "星期", "时间"]
	
	    for arr in sorted_schedule:
	        time_str = f"{time_slots[arr.time_slot]['start']}-{time_slots[arr.time_slot]['end']}"
	        table.add_row([arr.class_name, courses[arr.course_id],
	                       teachers[arr.teacher_id]['name'],
	                       arr.classroom, days[arr.day - 1], time_str])
	
	    return table
	
	
	# 主函数
	if __name__ == "__main__":
	    best_schedule, max_fitness, max_fitness_individual = genetic_algorithm()
	    best_fitness = calculate_fitness(best_schedule)  # 计算最佳排课方案的适应度值
	
	    # 确保最终答案的适应度不小于过程中最大适应度
	    if best_fitness < max_fitness:
	        best_schedule = max_fitness_individual
	        best_fitness = max_fitness
	
	    print("\n最佳排课方案:")
	    schedule_table = format_schedule(best_schedule)
	    print(schedule_table)
	    print(f"\n最佳适应度值:{best_fitness:.4f}")  # 打印最佳适应度值
	
	    # 将表格和适应度值保存到文件
	    with open('scheduling_result.txt', 'w') as f:
	        f.write(str(schedule_table))
	        f.write(f"\n最佳适应度值:{best_fitness:.4f}")
	\end{pythoncode}
\end{document}